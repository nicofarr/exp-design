\documentclass[xcolor=dvipsnames,english]{beamer}
\usepackage[utf8]{inputenc}
\usepackage[english]{babel}
\usepackage[T1]{fontenc}
\usepackage{amsfonts}
\usepackage{amsmath}
\usepackage{multirow}
\usepackage{amssymb}
\usepackage{tikz}
\usepackage[usestackEOL]{stackengine}
\usepackage{multirow}
\usepackage{graphicx,tabularx}
\usepackage{color}
\usepackage{pgfpages}
\usepackage[defaultsans]{cantarell}
\usepackage{graphicx}
\usepackage{tikz}
\usepackage{makecell}
\usepackage[usestackEOL]{stackengine}
\usepackage{pgfplots}
\usepackage[lofdepth,lotdepth]{subfig}
\usepackage{xspace}
\usetikzlibrary{arrows}
\usepackage{media9}

\definecolor{imtatlantique}{rgb}{0.005,0.715,0.867}
\usecolortheme[named=imtatlantique]{structure}
\setbeamercolor{normal text}{bg=white,fg=black}
\mode<presentation> {
  \usetheme{Madrid}
  \setbeamercovered{transparent}
}
\setbeamertemplate{items}[square]
\setbeamertemplate{section in toc}[square]

%\AtBeginSection[]{\begin{frame}<beamer>{Plan}\addtocounter{framenumber}{-1}\tableofcontents[currentsection]\end{frame}}
\makeatletter
\providecommand{\beamer@slideinframe}{0}%
\tikzset{highlight/.code={\ifnum#1=\beamer@slideinframe \tikzset{draw=orange,text=orange}\fi},highlight/.value required}
\makeatother
\definecolor{aquamarine}{rgb}{0,0.9,1}
\definecolor{orange}{rgb}{1,0.65,0.12}
\definecolor{salmon}{rgb}{1,0.5,0.5}
\definecolor{cyan}{rgb}{0,0.7,1}
\author[IMT-Atlantique]{Nicolas Farrugia}
\institute[]{\includegraphics[width=3cm]{Fig/logoimt.png}\hspace{5mm}\includegraphics[width=3cm]{Fig/logobrain.pdf}}
\date{UE SHS, Neurosciences et Musique, 2024\\ \url{https://github.com/nicofarr/exp-design}}

\title[Experimental design]{Introduction à la conception expérimentale}

\begin{document}
\pgfplotsset{compat=1.16}
\begin{frame}
  \maketitle
\end{frame}


\begin{frame}
  \frametitle{Séances de l'UE}
  \begin{enumerate}
  \item 16/01 matin : Neurosciences et neuromythes (D. Toquet).\\ après-midi : Pratique Musicale (C. Rocher)
  \item \alert{23/01 matin : Introduction à la conception expérimentale (N. Farrugia).} \\ après-midi : Pratique Musicale (C. Rocher)
  \item 30/01 Expériences EEG
  \item 06/02 matin : Analyse de données EEG (N. Farrugia). \\après-midi : Pratique Musicale (C. Rocher)
  \item 13/02 soir : restitution au Cabaret Vauban 
  \end{enumerate}
  \end{frame}

  

  \section{Contexte}


\begin{frame}
  \frametitle{Plan}
  
  \only<1>{\tableofcontents}
  \only<2>{\tableofcontents[currentsection]}
  

\end{frame}



% \subsection{Cognitive Neurosciences and Engineering}
% \begin{frame}
% \frametitle{Cognitive Neurosciences and Engineering}
% \begin{itemize}
% \item<1-> Cognitive neurosciences is the scientific field that is concerned with the study of the biological processes and aspects that underlie cognition, with a specific focus on the neural connections in the brain which are involved in mental processes.
% \item <2-> It addresses the questions of how cognitive activities are affected or controlled by neural circuits in the brain.
% \end{itemize}
% \end{frame}

% \subsection{What is experimental design?}

% \begin{frame}
% \frametitle{What is experimental design?}
% \begin{itemize}
% \item<1-> An experimental design is a planned experiment set up to test the validity of a hypothesis.
% \item <2-> An experimental design is a way to carefully plan experiments in advance.
% \end{itemize}
% \end{frame}


\subsection{Musique et Neurosciences}

\begin{frame}
  \frametitle{Neuroscience et musique}
  \only<1>{
    \begin{center}
          \includegraphics[width=0.8\textwidth]{Fig/whymusic2.png}
    \end{center}
    }
    \only<2>{
    La musique stimule un vaste réseau cérébral
    \begin{center}
          \includegraphics[width=0.8\textwidth]{Fig/zatorre.jpg}
    \end{center}
    
    }
    \only<3>{
    Musique et plasticité
    \begin{center}
          \includegraphics[width=0.8\textwidth]{Fig/musictraining.png}
    \end{center}
    \tiny{Groussard et al. 2014, Brain and Cognition. See also Giacosa et al. 2016, Neuroimage}
    }

\end{frame}


\subsection{Méthodes en neurosciences cognitives}

\begin{frame}
  \frametitle{Méthodes}

  \only<1>
  {
    \begin{center}
            \includegraphics[width=0.8\textwidth]{Fig/fmri.jpg}
      \end{center}
  }

\only<2>{

\begin{center}
          \includegraphics[width=0.8\textwidth]{Fig/eeg.jpg}
    \end{center}
}

\end{frame}


\subsection{Electroencéphalographie (EEG)}
\begin{frame}
  \frametitle{Electroencéphalographie (EEG)}

\only<1>{
Une vue d'ensemble de l'origine (probable) de l'EEG.

\begin{center}
\includegraphics[width=\textwidth]{images/baillet.png}
\end{center}
\tiny{Baillet, Sylvain, John C. Mosher, and Richard M. Leahy. “Electromagnetic brain mapping.” IEEE Signal processing magazine 18.6 (2001): 14-30.}
}

\only<2>{

\begin{columns}
 \column{.5\linewidth}
 Quelques caractéristiques clés de l'EEG
 	\begin{itemize}
 		\item Tension (V) entre une électrode sur le scalp et une référence
    \item Propagation du champ électrique dans plusieurs milieux
 		\item Signal résultant de l'activité coordonnée de groupes de millons de neurones
 		\item De 0.1 à 100 Hz, de l'ordre de 0.025 à 0.1 mV 
 		\item Très sensible aux artefacts
 	\end{itemize}


 \column{.5\linewidth}
\begin{center}
\includegraphics[width=0.7\textwidth]{images/nunez.jpg}
\end{center}


\end{columns}

}
\end{frame}


\begin{frame}
  \frametitle{Acquisition et prétraitement de l'EEG}


\only<1>{

\begin{columns}
\column{0.5\linewidth}
Choix d'une référence:
\begin{itemize}
  \item L'EEG mesure une différence de potentiel entre deux points
  \item Idéal (mythe!) : référence "silencieuse" (cf. Nunez)
  \item Nasion, mastoïdes, lobes, Cz, moyenne des électrodes, \dots
\end{itemize}


\column{0.5\linewidth}
\begin{center}
\includegraphics[width=0.7\textwidth]{images/eeg_ref.png}
\end{center}

\end{columns}


}

\only<2>{

\begin{columns}
\column{0.5\linewidth}
Le système 10-20 de placement d'électrodes:
%\begin{itemize}
%  \item Placement standards des électrodes
%  \item Idéal (Mythe!) : choisir une référence "silencieuse"
%  \item Nasion, mastoïdes, lobes, Cz, moyenne des électrodes, \dots
%\end{itemize}

\begin{center}
\includegraphics[width=0.9\textwidth]{images/1020_2.png}
\end{center}

\column{0.5\linewidth}
\begin{center}
\includegraphics[width=0.9\textwidth]{images/1020.png}
\end{center}

\end{columns}


}


\only<3>{
Exemple d'enregistrement

\begin{center}
\includegraphics[width=0.7\textwidth]{images/eeg_exemple.jpg}
\end{center}

\tiny{Electroencephalography (EEG): An Introductory Text and Atlas of Normal and Abnormal Findings in Adults, Children, and Infants}


}



\only<4-5>{
Prétraitements usuels
\begin{enumerate}
  \item Re-référencement
  \item Suppression de segments largement bruités
  \item Suppression de dérive de basse fréquence (filtre passe-haut)
  \alert<5>{\item Détection et suppression d'artefacts}
\end{enumerate}
\begin{center}
\includegraphics[width=0.7\textwidth]{images/eeg_exemple_2.jpg}
\end{center}

\tiny{\url{FrontalCortex.com}}


}



\only<6>{
Artefacts 

\begin{center}
\includegraphics[width=0.5\textwidth]{images/eeg_bio_artefacts.png}

\includegraphics[width=0.4\textwidth]{images/eeg_artefacts_2.png}
\end{center}

\tiny{Review of Artifact Rejection Methods for Electroencephalographic Systems, S Kanoga, Y Mitsukura - Electroencephalography, 2017}


}

\end{frame}


\section{Fondements de la conception expérimentale}
\subsection{Notions clés}


\begin{frame}
  \frametitle{Plan}
  \tableofcontents[currentsection]

\end{frame}

\begin{frame}{Cours de référence}
  Cours de Caroline Appert, Laboratoire de Recherche en Informatique, Université Paris-Sud


  \url{www.lri.fr/~appert/eval/classes/1-experimental-design.pdf}
\end{frame}

\begin{frame}{Qu'est ce qu'une expérience ? }

\begin{block}{Expérience}
  Un test fait pour démontrer une vérité connue, examiner la validité d'une hypothèse, ou déterminer l'efficacité de quelque chose qui n'a pas encore été essayé.

\end{block}

\uncover<2>{\begin{block}{Conditions}
  Elle est conduite dans des conditions hautement contrôlées (pas nécessairement en laboratoire), où des mesures précises sont possibles. Le chercheur décide où l'expérience aura lieu, à quel moment, avec quels participants, dans quelles circonstances et en utilisant une procédure standardisée.
\end{block}}

\tiny{Source : \url{https://www.thefreedictionary.com/} et \url{https://www.simplypsychology.org/}}


\end{frame}

\begin{frame}{Hypothèse}
  
  \begin{block}{Hypothèse}
   Une supposition ou une explication proposée faite sur la base de preuves limitées comme point de départ pour une enquête plus approfondie
  \end{block}

  \uncover<2>{Propriétés d'une hypothèse :
  \begin{itemize}
    \item Elle doit être testable : on doit pouvoir en manipuler les facteurs et/ou mesurer la variable dépendante
    \item Elle doit être falsifiable : il doit être possible de la réfuter avec des données
    \item Elle doit être précise : on doit pouvoir en définir les termes et conditions de manière spécifique (opéralionalisée).
  \end{itemize}}


\tiny{\url{https://www.lri.fr/~appert/eval/classes/1-experimental-design.pdf}}
\end{frame}

\begin{frame}
  \frametitle{Tester une hypothèse}
  L'expérimentateur manipule des variables indépendantes et mesure des variables dépendantes.
  \begin{block}{Variables}
    \begin{itemize}
      \item Variable \textit{indépendante} = facteur que l'on teste, doit être manipulable
      \item<2-> Variable \textit{dépendante} = mesure effectuée, doit être observable
      \item<3-> \textit{Covariée} = variable qui peut influencer la variable dépendante, doit être contrôlée (ex : âge, sexe, \dots) ou son influence doit être limitée (ex : randomisation)
    \end{itemize}
  \end{block}
  \uncover<4>{
  \begin{block}{Groupes}
    \begin{itemize}
      \item Groupe expérimental
      \item Groupe contrôle
    \end{itemize}
  \end{block}
  }

\end{frame}

\subsection{Types de conception expérimentale}

\begin{frame}
  \frametitle{Types de conception expérimentale}

  \begin{block}{Conception entre-sujets}
    \begin{itemize}
      \item<1-> Chaque sujet ne participe qu'à une seule condition expérimentale
      \item<2-> Exemple : 2 groupes de sujets, un groupe écoute de la musique, l'autre non
    \end{itemize}
  \end{block}

  \begin{block}{Conception intra-sujets}
    \begin{itemize}
      \item<3-> Chaque sujet participe à toutes les conditions expérimentales
      \item<4-> Exemple : chaque sujet écoute de la musique à certains moments, puis ne l'écoute pas à d'autres.
    \end{itemize}
  \end{block}
\end{frame}

\section{Examples}
\subsection{Reading time ! Cognitive Neurosciences studies}

\begin{frame}
  \frametitle{Articles}
\end{frame}

\subsection{Discussion}

\begin{frame}
  \frametitle{Discussion}
\end{frame}


\section{Experimental design in practice}
\subsection{Practice time ! Design your own experiment}

\begin{frame}
  \frametitle{Design your own experiment}
\end{frame}

\subsection{Presentations / Discussion}

\begin{frame}
  \frametitle{Presentations / Discussion}
\end{frame}


\section{Conclusion}
\subsection{Wrap-up}

\begin{frame}
  \frametitle{Wrap-up}
\end{frame}

\subsection{What's next ?}



\begin{frame}
  \frametitle{Et ensuite ?}
  Séances : 
  \begin{enumerate}
  \item 16/01 matin : Neurosciences et neuromythes (D. Toquet).\\ après-midi : Pratique Musicale (C. Rocher)
  \item 23/01 matin : Introduction à la conception expérimentale (N. Farrugia). \\ après-midi : Pratique Musicale (C. Rocher)
  \item 30/01 Expériences EEG
  \item 06/02 matin : Analyse de données EEG (N. Farrugia). \\après-midi : Pratique Musicale (C. Rocher)
  \item 13/02 soir : restitution au Cabaret Vauban 
  \end{enumerate}
  \end{frame}
% \begin{frame}
% \frametitle{Introduction}
% \begin{itemize}
% \item<1-> What is an experiment?
% \item <2-> What is an experimental design?
% \item <3-> What is a control group?
% \item <4-> What is a placebo?
% \item <5-> What is a double-blind experiment?
% \item <6-> What is a single-blind experiment?
% \item <7-> What is a between-subjects experiment?
% \item <8-> What is a within-subjects experiment?
% \item <9-> What is a mixed design experiment?
% \item <10-> What is a counterbalanced experiment?
% \item <11-> What is a repeated measures experiment?
% \item <12-> What is a matched pairs experiment?
% \item <13-> What is a factorial experiment?
% \item <14-> What is a confounding variable?
% \item <15-> What is a covariate?
% \item <16-> What is a random variable?
% \item <17-> What is a random effect?
% \item <18-> What is a fixed effect?
% \end{itemize}
% \end{frame}


% \section{What is an experiment?}
% \begin{frame}
% \frametitle{What is an experiment?}
% \begin{itemize}
% \item<1-> An experiment is a procedure carried out to support, refute, or validate a hypothesis.
% \item <2-> Experiments provide insight into cause-and-effect by demonstrating what outcome occurs when a particular factor is manipulated.
% \item <3-> Experiments vary greatly in their goal and scale, but always rely on repeatable procedure and logical analysis of the results.
% \item <4-> There also exists natural experimental studies.
% \item <5-> In an experiment, an independent variable (the cause) is manipulated and the dependent variable (the effect) is measured; any extraneous variables are controlled.
% \item <6-> An experiment is a type of research method in which you manipulate one or more independent variables and measure their effect on one or more dependent variables.
% \item <7-> The goal of an experiment is to determine whether the independent variable causes a change in the dependent variable.
% \end{itemize}
% \end{frame}

% \section{What is an experimental design?}
% \begin{frame}
% \frametitle{What is an experimental design?}
% \begin{itemize}
% \item<1-> An experimental design is a planned experiment set up to test the validity of a hypothesis.
% \item <2-> An experimental design is a way to carefully plan experiments in advance.
% \item <3-> Types of experimental design:
% \item <4-> \begin{itemize}
% \item <4-> Between-subjects design
% \item <5-> Within-subjects design
% \item <6-> Mixed design
% \end{itemize}
% \end{itemize}
% \end{frame}

% \section{What is a control group?}
% \begin{frame}
% \frametitle{What is a control group?}
% \begin{itemize}
% \item<1-> A control group is a group separated from the rest of the experiment such that the independent variable being tested cannot influence the results.
% \item <2-> This isolates the independent variable's effects on the experiment and can help rule out alternative explanations of the experimental results.
% \item <3-> A control group can be identified by:
% \item <4-> \begin{itemize}
% \item <4-> Placebo effect
% \item <5-> Hawthorne effect
% \item <6-> Experimenter's bias
% \item <7-> Demand characteristics
% \item <8-> Order effects
% \item <9-> Practice effects
% \item <10-> Fatigue effects
% \item <11-> Carry-over effects
% \item <12-> Confounding variables
% \end{itemize}
% \end{itemize}
% \end{frame}




%Dernière frame numérotée avant ceci
\appendix
\newcounter{finalframe}
\setcounter{finalframe}{\value{framenumber}}

%%%%%%%%%%%%%%%%%%%%%%%%%%%%%%%%%%%%%%%%%%%%%%%%%%%%%%%%%%%%%%%%%%%%%%%%%%%%%%%%%%%%%%%%%%%%%%%%%%
%%%%%%%%%%%%%%%%%%%%%%%%%%%%%%%%%%%%%%%%%%%%%%%%%%%%%%%%%%%%%%%%%%%%%%%%%%%%%%%%%%%%%%%%%%%%%%%%%%
%%%%%%%%%%%%%%%%%%%%%%%%%%%%%%%%%%%%%%%%%%%%%%%%%%%%%%%%%%%%%%%%%%%%%%%%%%%%%%%%%%%%%%%%%%%%%%%%%%
%%%%%%%%%%%%%%%%%%%%%%%%%%%%%%%%%%%%%%%%%%%%%%%%%%%%%%%%%%%%%%%%%%%%%%%%%%%%%%%%%%%%%%%%%%%%%%%%%%
%%%%%%%%%%%%%%%%%%%%%%%%%%%%%%%%%%%%%%%%%%%%%%%%%%%%%%%%%%%%%%%%%%%%%%%%%%%%%%%%%%%%%%%%%%%%%%%%%%
%%%%%%%%%%%%%%%%%%%%%%%%%%%%%%%%%%%%%%%%%%%%%%%%%%%%%%%%%%%%%%%%%%%%%%%%%%%%%%%%%%%%%%%%%%%%%%%%%%
%%%%%%%%%%%%%%%%%%%%%%%%%%%%%%%%%%%%%%%%%%%%%%%%%%%%%%%%%%%%%%%%%%%%%%%%%%%%%%%%%%%%%%%%%%%%%%%%%%
%%%%%%%%%%%%%%%%%%%%%%%%%%%%%%%%%%%%%%%%%%%%%%%%%%%%%%%%%%%%%%%%%%%%%%%%%%%%%%%%%%%%%%%%%%%%%%%%%%
%%%%%%%%%%%%%%%%%%%%%%%%%%%%%%%%%%%%%%%%%%%%%%%%%%%%%%%%%%%%%%%%%%%%%%%%%%%%%%%%%%%%%%%%%%%%%%%%%%
%%%%%%%%%%%%%%%%%%%%%%%%%%%%%%%%%%%%%%%%%%%%%%%%%%%%%%%%%%%%%%%%%%%%%%%%%%%%%%%%%%%%%%%%%%%%%%%%%%
%%%%%%%%%%%%%%%%%%%%%%%%%%%%%%%%%%%%%%%%%%%%%%%%%%%%%%%%%%%%%%%%%%%%%%%%%%%%%%%%%%%%%%%%%%%%%%%%%%
%%%%%%%%%%%%%%%%%%%%%%%%%%%%%%%%%%%%%%%%%%%%%%%%%%%%%%%%%%%%%%%%%%%%%%%%%%%%%%%%%%%%%%%%%%%%%%%%%%
%%%%%%%%%%%%%%%%%%%%%%%%%%%%%%%%%%%%%%%%%%%%%%%%%%%%%%%%%%%%%%%%%%%%%%%%%%%%%%%%%%%%%%%%%%%%%%%%%%
%%%%%%%%%%%%%%%%%%%%%%%%%%%%%%%%%%%%%%%%%%%%%%%%%%%%%%%%%%%%%%%%%%%%%%%%%%%%%%%%%%%%%%%%%%%%%%%%%%
%%%%%%%%%%%%%%%%%%%%%%%%%%%%%%%%%%%%%%%%%%%%%%%%%%%%%%%%%%%%%%%%%%%%%%%%%%%%%%%%%%%%%%%%%%%%%%%%%%
%%%%%%%%%%%%%%%%%%%%%%%%%%%%%%%%%%%%%%%%%%%%%%%%%%%%%%%%%%%%%%%%%%%%%%%%%%%%%%%%%%%%%%%%%%%%%%%%%%
%%%%%%%%%%%%%%%%%%%%%%%%%%%%%%%%%%%%%%%%%%%%%%%%%%%%%%%%%%%%%%%%%%%%%%%%%%%%%%%%%%%%%%%%%%%%%%%%%%
%%%%%%%%%%%%%%%%%%%%%%%%%%%%%%%%%%%%%%%%%%%%%%%%%%%%%%%%%%%%%%%%%%%%%%%%%%%%%%%%%%%%%%%%%%%%%%%%%%
%%%%%%%%%%%%%%%%%%%%%%%%%%%%%%%%%%%%%%%%%%%%%%%%%%%%%%%%%%%%%%%%%%%%%%%%%%%%%%%%%%%%%%%%%%%%%%%%%%
%%%%%%%%%%%%%%%%%%%%%%%%%%%%%%%%%%%%%%%%%%%%%%%%%%%%%%%%%%%%%%%%%%%%%%%%%%%%%%%%%%%%%%%%%%%%%%%%%%
%%%%%%%%%%%%%%%%%%%%%%%%%%%%%%%%%%%%%%%%%%%%%%%%%%%%%%%%%%%%%%%%%%%%%%%%%%%%%%%%%%%%%%%%%%%%%%%%%%
%%%%%%%%%%%%%%%%%%%%%%%%%%%%%%%%%%%%%%%%%%%%%%%%%%%%%%%%%%%%%%%%%%%%%%%%%%%%%%%%%%%%%%%%%%%%%%%%%%
%%%%%%%%%%%%%%%%%%%%%%%%%%%%%%%%%%%%%%%%%%%%%%%%%%%%%%%%%%%%%%%%%%%%%%%%%%%%%%%%%%%%%%%%%%%%%%%%%%
%%%%%%%%%%%%%%%%%%%%%%%%%%%%%%%%%%%%%%%%%%%%%%%%%%%%%%%%%%%%%%%%%%%%%%%%%%%%%%%%%%%%%%%%%%%%%%%%%%
%%%%%%%%%%%%%%%%%%%%%%%%%%%%%%%%%%%%%%%%%%%%%%%%%%%%%%%%%%%%%%%%%%%%%%%%%%%%%%%%%%%%%%%%%%%%%%%%%%
%%%%%%%%%%%%%%%%%%%%%%%%%%%%%%%%%%%%%%%%%%%%%%%%%%%%%%%%%%%%%%%%%%%%%%%%%%%%%%%%%%%%%%%%%%%%%%%%%%
%%%%%%%%%%%%%%%%%%%%%%%%%%%%%%%%%%%%%%%%%%%%%%%%%%%%%%%%%%%%%%%%%%%%%%%%%%%%%%%%%%%%%%%%%%%%%%%%%%
%%%%%%%%%%%%%%%%%%%%%%%%%%%%%%%%%%%%%%%%%%%%%%%%%%%%%%%%%%%%%%%%%%%%%%%%%%%%%%%%%%%%%%%%%%%%%%%%%%
%%%%%%%%%%%%%%%%%%%%%%%%%%%%%%%%%%%%%%%%%%%%%%%%%%%%%%%%%%%%%%%%%%%%%%%%%%%%%%%%%%%%%%%%%%%%%%%%%%
%%%%%%%%%%%%%%%%%%%%%%%%%%%%%%%%%%%%%%%%%%%%%%%%%%%%%%%%%%%%%%%%%%%%%%%%%%%%%%%%%%%%%%%%%%%%%%%%%%
%%%%%%%%%%%%%%%%%%%%%%%%%%%%%%%%%%%%%%%%%%%%%%%%%%%%%%%%%%%%%%%%%%%%%%%%%%%%%%%%%%%%%%%%%%%%%%%%%%
%%%%%%%%%%%%%%%%%%%%%%%%%%%%%%%%%%%%%%%%%%%%%%%%%%%%%%%%%%%%%%%%%%%%%%%%%%%%%%%%%%%%%%%%%%%%%%%%%%
%%%%%%%%%%%%%%%%%%%%%%%%%%%%%%%%%%%%%%%%%%%%%%%%%%%%%%%%%%%%%%%%%%%%%%%%%%%%%%%%%%%%%%%%%%%%%%%%%%
%%%%%%%%%%%%%%%%%%%%%%%%%%%%%%%%%%%%%%%%%%%%%%%%%%%%%%%%%%%%%%%%%%%%%%%%%%%%%%%%%%%%%%%%%%%%%%%%%%
%%%%%%%%%%%%%%%%%%%%%%%%%%%%%%%%%%%%%%%%%%%%%%%%%%%%%%%%%%%%%%%%%%%%%%%%%%%%%%%%%%%%%%%%%%%%%%%%%%
%%%%%%%%%%%%%%%%%%%%%%%%%%%%%%%%%%%%%%%%%%%%%%%%%%%%%%%%%%%%%%%%%%%%%%%%%%%%%%%%%%%%%%%%%%%%%%%%%%
%%%%%%%%%%%%%%%%%%%%%%%%%%%%%%%%%%%%%%%%%%%%%%%%%%%%%%%%%%%%%%%%%%%%%%%%%%%%%%%%%%%%%%%%%%%%%%%%%%
%%%%%%%%%%%%%%%%%%%%%%%%%%%%%%%%%%%%%%%%%%%%%%%%%%%%%%%%%%%%%%%%%%%%%%%%%%%%%%%%%%%%%%%%%%%%%%%%%%
%%%%%%%%%%%%%%%%%%%%%%%%%%%%%%%%%%%%%%%%%%%%%%%%%%%%%%%%%%%%%%%%%%%%%%%%%%%%%%%%%%%%%%%%%%%%%%%%%%
%%% Backup slides 


%Juste avant la fin du document
\setcounter{framenumber}{\value{finalframe}}

\end{document}
